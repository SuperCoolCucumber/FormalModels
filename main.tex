% !TEX program = lualatex
\documentclass[a4paper,12pt]{article}
\usepackage[document]{ragged2e}
\usepackage{adjustbox}
\usepackage{fontspec}
\usepackage{pifont}
\setlength{\parskip}{\baselineskip}
\setlength{\parindent}{0pt}

% The default Latin Modern font family doesn't contain Cyrillics,
% so load Computer Modern Unicode instead
\setmainfont{CMU Serif}[Ligatures=TeX]

\usepackage[english,russian]{babel}

%%%%%%%%%%%%%
% \setlength{\parskip}{\baselineskip}
% \setlength{\parindent}{0pt}



\author{Дарья Галимзянова}
\title{Формальное представление естественного языка}
\date{\today}

\begin{document}

\maketitle
\textbf{1. Глубинные представления. Генеративная фонология}

\textbf{1.1. Если считать глубинные представления (ГП) суффиксов такими: /i/, /no/, /go/,
/do/, то какими должны быть глубинные представления слов ‘окно’, ‘ягода’, ‘её
окно’, ‘её ягода’? Почему?}


Глубнными представлениями можно считать следующие:

Окно - 'bugat'\\
Ягода - 'tugad'\\
Ее окно - 'bugat-do'\\
Ее ягода - 'tugad-do'

Судя по всему, в языке тангале существует правило озвончения согласного аффикса после последнего звонкого звука в основе слова. Так, например, в ПП мы видим 'kuluk' и 'kulug-no', 'kulug-go' и т.д. Однако, видимо, еще существует правило не озвончать согласный перед аффиксом '-i', как мы это видим в примерах 'aduk' - 'aduk-i' и 'bugat' - 'bugat-i'. Мы руководствуемся этими правилами, чтобы обнаружить глубинное представление выше указанных слов. Поскольку ГП -- это форма слова до применения правил, 'ее окно' имеет ГП в виде 'bugat-do', а не 'bugad-do'.

\textbf{1.2. Почему предложенные в п.1 ГП /go/ и /do/ кажутся более адекватными, чем /ko/
и /to/?}

Если мы будем считать суффиксы /ko/ и /to/ глубинными представлениями, то нам понадобится правило, объясняющее, что после гласной в корне (ПП 'loo-do') и после звонкой согласной (ПП 'tugad-do') аффикс озвончается.

\textbf{1.3. Кажется, в тангале представлена прогрессивная и регрессивная ассимиляция по
глухости/звонкости. Сформулируйте два необходимых правила, A и B. Используйте
спецификации $\alpha$ voice для признака, значение которого совпадает в фокусе и в
контексте правила.}

Правило А (для регрессивного озвончения):\\
$[-syll, +cons, -voice] \rightarrow [-syll, +cons, +voice] \_n\ $

Правило B (для прогрессивного оглушения):\\
$[-syll, +cons, +voice] ️\rightarrow [\alpha voice]/[\alpha voice]\_$

\textbf{Определите порядок применения правил A и B. Почему порядок должен быть
именно таким? Если порядок не важен, тоже напишите почему.}

Порядок применения правил не важен

\textbf{Приведите деривации слов bugadno и bugatko в следующем формате:}

\begin{center}
\begin{tabular}{|c c c|} 
 \hline
  & моё окно & твоё окно \\ [0.5ex] 
 \hline
 ГП & /bugat-no/ & /bugat-go/ \\ 
 \hline
 Правило X & bugad-no & bugat-go \\
 \hline
 Правило Y & bugad-no & bugat-ko \\
 \hline
 ПП & [bugad-no] & [bugat-ko] \\
 \hline
\end{tabular}
\end{center}

\textbf{1.4. Для анализа фрагмента парадигмы, приведенного выше, требуется еще одно
правило (C). Сформулируйте это правило и приведите два примера его применения.}

Правило C: \\
$[+cons, +voice] \rightarrow [-voice]/ \_ \# $ \\

Пример 1:\\
$tugad \rightarrow tugat$ \\
В парадигме есть слова aduk и kuluk, в которых последняя согласная не одинакова в ГП и ПП. Чтобы избавиться от этого несоответствия, мы можем предложить правило С, оглушающее последнюю согласную в конце ГП.

Пример 2:\\
$kulug \rightarrow kuluk$ \\

\textbf{1.5 Определите порядок применения имеющихся правил (релевантны три правила из имеющихся четырех). Аргументируйте ваше решение. Приведите деривации слов
‘моя соль’ и ‘твоя соль’:}

Правило С из имеющихся применяется только на конце слова, все остальные модифицируют морфемы, находящиеся внутри слова. Таким образом, мы используем только правила A, B и D.

\begin{center}
\begin{tabular}{|c c c|} 
 \hline
  & моя соль & твоя соль \\ [0.5ex] 
 \hline
 ГП & /duka-no/ & /duka-ko/ \\ 
 \hline
 Правило B & duka-no & duka-go \\
 \hline
 Правило D & duk-no & duk-ko \\
 \hline
 Правило A & duk-no & duk-ko \\
 \hline
 ПП & [duk-no] & [duk-ko] \\
 \hline
\end{tabular}
\end{center}

\textbf{1.6*. Какое из правил можно считать «непрозрачным»? Мы имеем здесь дело с
underapplication (=counterfeeding) или overapplication (=counterbleeding)? Поясните ваш
ответ.}

Возможно, непрозрачным может считаться правило D, так как при его отмене мы не можем применить правило А. В этом случае мы имеем дело с underapplication (=counterfeeding), то есть с нарушением следования правил.

\textbf{2. Теория оптимальности: ограничения в тангале}

\textbf{2.1. Как ранжированы $AGREE(CC)^{[voice]}, IDENT([voice])_{aff},  IDENT([voice])_{root}$? (Данные совместимы с двумя из шести возможных полных ранжирований
трех ограничений). Покажите, как выбирается оптимальное ПП с помощью
таблички (tableau), в которой для ГП /bugatgo/ соревнуются потенциальные
ПП [bugatgo], [bugadgo] и [bugatko].}

\begin{center}
\begin{tabular}{|c|c|c|c|} 
 \hline
 /bugat-go/ & $AGREE(CC)^{[voice]}$ & $IDENT([voice])_{aff}$ & $IDENT([voice])_{root}$ \\ [0.5ex] 
 \hline
 bugat-go & * & & \\ 
 \hline
 bugad-go & & * & \\
 \hline
 \ding{228} bugat-ko & & & * \\
 \hline
\end{tabular}
\end{center}



\textbf{2.2. Для каждого из неправильных ранжирований этих трех ограничений укажите
кандидата, который будет при таком ранжировании побеждать.}


\begin{center}
\begin{tabular}{|c|c|c|c|} 
 \hline
 /bugat-go/ & $IDENT([voice])_{aff}$ & $IDENT([voice])_{root}$ & $AGREE(CC)^{[voice]}$ \\ [0.5ex] 
 \hline
 \ding{228} bugat-go & & & * \\ 
 \hline
 bugad-go & & * & \\
 \hline
 bugat-ko & * & & \\
 \hline
\end{tabular}
\end{center}


\begin{center}
\begin{tabular}{|c|c|c|c|} 
 \hline
 /bugat-go/ & $IDENT([voice])_{root}$ & $AGREE(CC)^{[voice]}$ & $IDENT([voice])_{aff}$ \\ [0.5ex] 
 \hline
 bugat-go & & * & \\ 
 \hline
 bugad-go & * & & \\
 \hline
 \ding{228} bugat-ko & & & * \\
 \hline
\end{tabular}
\end{center}


\begin{center}
\begin{tabular}{|c|c|c|c|} 
 \hline
 /bugat-go/ & $IDENT([voice])_{aff}$ & $AGREE(CC)^{[voice]}$ & $IDENT([voice])_{root}$ \\ [0.5ex] 
 \hline
 bugat-go & & * & \\ 
 \hline
 \ding{228} bugad-go & & & * \\
 \hline
 bugat-ko & * & & \\
 \hline
\end{tabular}
\end{center}

\begin{center}
\begin{tabular}{|c|c|c|c|} 
 \hline
 /bugat-go/ & $IDENT([voice])_{root}$ & $IDENT([voice])_{aff}$ & $AGREE(CC)^{[voice]}$ \\ [0.5ex] 
 \hline
 \ding{228} bugat-go & & & * \\ 
 \hline
 bugad-go & * & & \\
 \hline
 bugat-ko & & * & \\
 \hline
\end{tabular}
\end{center}



\begin{center}
\begin{tabular}{|c|c|c|c|} 
 \hline
 /bugat-go/ & $AGREE(CC)^{[voice]}$ & $IDENT([voice])_{aff}$ & $IDENT([voice])_{root}$ \\ [0.5ex] 
 \hline
 bugat-go & * & & \\ 
 \hline
 \ding{228} bugad-go & & & * \\
 \hline
 bugat-ko & & * & \\
 \hline
\end{tabular}
\end{center}


\textbf{2.3. Почему присоединение суффикса, начинающегося на сонорный n, не приводит
к оглушению самого n, а вызывает регрессивную ассимиляцию последнего
согласного корня? Здесь может помочь еще одно ограничение маркированности,
некое ограничение X.}

Присоединение суффикса, начинаюегося на сонорный n, не приводит к его оглушению, потому что у сонорных нет парных глухих звуков.

Ограничение X: \\
$[–syll, +cons, –voice] → [\alpha voice] / \_ [–syll, +cons, +son, \alpha voice]$


\begin{center}
\begin{tabular}{|c|c|c|c|c|}
 \hline
 /bugat-no/ & $AG(CC)^{[voice]}$ & $AG(CC)^{[+voice]}$ & $ID([voice])_{aff}$ &  $ID([voice])_{root}$ \\ [0.5ex] 
 \hline
 bugat-no & * & & & \\ 
 \hline
 \ding{228} bugad-no & & & & * \\
 \hline
 bugat-no & & * & * & \\
 \hline
\end{tabular}
\end{center}


\textbf{2.4. Объясните, почему для /bugatno/ в качестве оптимального ПП не выбирается
[bugatto], встроив в иерархию еще одно ограничение.
Опишите это ограничение: это ограничение маркированности или верности? что
оно запрещает? Как именно оно встраивается в имеющуюся на данный момент
иерархию? Проиллюстрируйте итоговое ранжирование с помощью таблички с ГП
/bugatno/ и кандидатами [bugatno], [bugadno], [bugatNo] и [bugatto].}



\begin{tabular}{|c|c|c|c|c|c|}
 \hline
 /bugat-no/ & $AG(CC)^{[voice]}$ & $AG(CC)^{[+voi]}$ & $ID([voi])_{aff}$ &  $ID([voice])_{root}$ & $ID([son])_{aff}$ \\ [0.5ex] 
 \hline
 bugat-no & & * & & & \\ 
 \hline
 \ding{228} bugad-no & & & & * & \\
 \hline
 bugat-no & & * & * & & \\
 \hline
 bugat-to & & & * & & * \\
 \hline
\end{tabular}

\end{document}

